% This is the Reed College LaTeX thesis template. Most of the work
% for the document class was done by Sam Noble (SN), as well as this
% template. Later comments etc. by Ben Salzberg (BTS). Additional
% restructuring and APA support by Jess Youngberg (JY).
% Further modifications done by James Spencer to adapt to MacMaster University
% template.
%
% See http://web.reed.edu/cis/help/latex.html for help. There are a
% great bunch of help pages there, with notes on
% getting started, bibtex, etc. Go there and read it if you're not
% already familiar with LaTeX.
%
% Any line that starts with a percent symbol is a comment.
% They won't show up in the document, and are useful for notes
% to yourself and explaining commands.
% Commenting also removes a line from the document;
% very handy for troubleshooting problems. -BTS

% As far as I know, this follows the requirements laid out in
% the 2002-2003 Senior Handbook. Ask a librarian to check the
% document before binding. -SN

%%
%% Preamble
%%
% \documentclass{<something>} must begin each LaTeX document
\documentclass[12pt,twoside]{Mactemplate}
% Packages are extensions to the basic LaTeX functions. Whatever you
% want to typeset, there is probably a package out there for it.
% Chemistry (chemtex), screenplays, you name it.
% Check out CTAN to see: http://www.ctan.org/
%%
\usepackage{graphicx,latexsym}
\usepackage{amsmath}
\usepackage{amssymb,amsthm}
\usepackage{longtable,booktabs,setspace}
\doublespacing
\usepackage{chemarr} %% Useful for one reaction arrow, useless if you're not a chem major
\usepackage[hyphens]{url}
% Added by CII
\usepackage{hyperref}
\usepackage{lmodern}
\usepackage{float}
\floatplacement{figure}{H}
% End of CII addition
\usepackage{rotating}

% Next line commented out by CII
%%% \usepackage{natbib}
% Comment out the natbib line above and uncomment the following two lines to use the new
% biblatex-chicago style, for Chicago A. Also make some changes at the end where the
% bibliography is included.
%\usepackage{biblatex-chicago}
%\bibliography{thesis}


% Added by CII (Thanks, Hadley!)
% Use ref for internal links
\renewcommand{\hyperref}[2][???]{\autoref{#1}}
\def\chapterautorefname{Chapter}
\def\sectionautorefname{Section}
\def\subsectionautorefname{Subsection}
% End of CII addition

% Added by CII
\usepackage{caption}
\captionsetup{width=5in}
% End of CII addition

\usepackage{times} % other fonts are available like times, bookman, charter, palatino


% To pass between YAML and LaTeX the dollar signs are added by CII
\title{Essays in International Trade and Finance}
\author{Stephanie M. Houle}
\authorshort{S Houle}
% The month and year that you submit your FINAL draft TO THE LIBRARY (May or December)


\advisor{Michael Veall and Pau Pujolas}
\committee{Gajendran Raveendranathan}
\institution{McMaster University}
\degree{Doctor of Philosophy}
\degreeshort{Ph.D.}
\education{B.A., M.A.}
\month{August}
\year{2018}
\city{Hamilton}
\province{Ontario}

% End of CII addition
%HALF TITLE
\halftitle{Essays in International Trade and Finance}

%DEPARTMENT
\department{Economics}


% Added by CII
%%% Copied from knitr
%% maxwidth is the original width if it's less than linewidth
%% otherwise use linewidth (to make sure the graphics do not exceed the margin)
\makeatletter
\def\maxwidth{ %
  \ifdim\Gin@nat@width>\linewidth
    \linewidth
  \else
    \Gin@nat@width
  \fi
}
\makeatother

\renewcommand{\contentsname}{Table of Contents}
% End of CII addition

\setlength{\parskip}{0pt}

% Added by CII

\providecommand{\tightlist}{%
  \setlength{\itemsep}{0pt}\setlength{\parskip}{0pt}}

\Layabstract{

}

\Abstract{
There has been a huge increase in the volume of trade in the last fifty
years, fueled in part by the proliferation of international agreements.
This thesis studies two important implications of these agreements as
well as a third implication of a comparison of specific patterns of
rising trade across countries. After an overview of the thesis in
Chapter 1, Chapter 2 examines detailed firm-level microdata on firms who
invested and traded with Peru, before and after the Canada-Peru Foreign
Investment Protection Agreement enacted in 2007. It finds little
evidence that the agreement contributed to outsourcing in Canada. It
also finds that in this case, the firm's Foreign Direct Investment was
more likely to have expanded their production structures horizontally
rather than vertically, although the evidence is incomplete. Chapter 3
uses a theoretical model to show potential shortcomings of reducing
tariffs through international agreements when governments may face a
sovereign debt crisis, especially when their institutions have limited
ability to collect other forms of taxes. Chapter 4 examines trade data
on imports of luxury goods, finding no robust evidence of different
rates of changes across countries that have been estimated by others to
have had large or small increases in top end incomes.
}


\Acknowledgements{
I would like to thank my supervisors, Michael Veall and Pau Pujolas, who
skillfully guided me through my research endeavours. I would also like
to thank Gajendran Raveendranathan, for his insights, especially in the
writing and computational programming for Chapter 3. Thank you to the
people at the Canadian Centre for Data Development and Economic Research
(CDER) for making the data used in Chapter 2 available for research and
for all the helpful comments I received during my time there. Thank you
as well to the people at the Department of Economics of the University
of Minnesota, especially Tim Kehoe who helped me devise the ideas for
Chapters 2 and 3 of my thesis. The knowledge and discussions I was able
to engage in during my time at the department proved invaluable to my
success.

I would like to acknowledge the funding received from the Productivity
Partnership as supported by the Social Sciences and Humanities Research
Council of Canada. It made it possible to access the data that was
critical in the analysis for Chapter 2.

Special thanks to others who have offered advice throughout my years as
a doctoral student. In no particular order, Professors Svetlana
Demidova, Cesar Sosa-Padilla, Arthur Sweetman, Alok Johri, Jeff Racine,
Bettina Brueggemann, Stephen Jones and Phil DeCicca. As well as students
Khuong Trong (the real HLP), Karen Bravo, Mahbub Rahman, Terry Yip,
Muhebullah Karimzada, Shaun Shaikh, Farhana Khanam, Sean Sexton, Luc
Clair, Grant Gibson, John Kealey, Natalie Malak, Nadine Chami, Anthony
Hong and Zvez Todorov. A special thank to James Spencer who, among many
other contributions, created this McMaster thesis template in \emph{R
Markdown} allowing for seamless formatting of this document.

Finally, I am thankful for the love and support of my parents and
grandfather who always encouraged my academic endeavours. Also, to my
aunts and uncles who have played a role in my life; Lise, Monique,
Jacques, Darcie, Danny and Lisa. And to my best friend Set.

This thesis is dedicated to my late mother Ginette Houle who taught me
about courage and tenacity and who supported my fascination with science
and mathematics at a young age.
}

\Declaration{
\label{doaac} The content of the research in this thesis is composed of
work by Stephanie Houle, Pau S. Pujolas and Michael R. Veall. Chapters 2
and 3 are solely the work of Stephanie Houle while Chapter 4 was joint
work with Pau S. Pujolas and Michael R. Veall. To Chapter 4, I
contributed all the data work, tables and graphs, and the writing for
Section 4.5.
}






% End of CII addition
%%
%% End Preamble
%%
%
\begin{document}

% Everything below added by CII
\frontmatter % this stuff will be roman-numbered
%\pagestyle{empty} % this removes page numbers from the frontmatter
%\pagestyle{fancyplain} %FOR PAGE NUMBERING
\pagenumbering{roman}
  \maketitle

%\frontmatter

  \begin{abstract}
    There has been a huge increase in the volume of trade in the last fifty
    years, fueled in part by the proliferation of international agreements.
    This thesis studies two important implications of these agreements as
    well as a third implication of a comparison of specific patterns of
    rising trade across countries. After an overview of the thesis in
    Chapter 1, Chapter 2 examines detailed firm-level microdata on firms who
    invested and traded with Peru, before and after the Canada-Peru Foreign
    Investment Protection Agreement enacted in 2007. It finds little
    evidence that the agreement contributed to outsourcing in Canada. It
    also finds that in this case, the firm's Foreign Direct Investment was
    more likely to have expanded their production structures horizontally
    rather than vertically, although the evidence is incomplete. Chapter 3
    uses a theoretical model to show potential shortcomings of reducing
    tariffs through international agreements when governments may face a
    sovereign debt crisis, especially when their institutions have limited
    ability to collect other forms of taxes. Chapter 4 examines trade data
    on imports of luxury goods, finding no robust evidence of different
    rates of changes across countries that have been estimated by others to
    have had large or small increases in top end incomes.
    \thispagestyle{plain}
  \end{abstract}
  \begin{acknowledgements}
    I would like to thank my supervisors, Michael Veall and Pau Pujolas, who
    skillfully guided me through my research endeavours. I would also like
    to thank Gajendran Raveendranathan, for his insights, especially in the
    writing and computational programming for Chapter 3. Thank you to the
    people at the Canadian Centre for Data Development and Economic Research
    (CDER) for making the data used in Chapter 2 available for research and
    for all the helpful comments I received during my time there. Thank you
    as well to the people at the Department of Economics of the University
    of Minnesota, especially Tim Kehoe who helped me devise the ideas for
    Chapters 2 and 3 of my thesis. The knowledge and discussions I was able
    to engage in during my time at the department proved invaluable to my
    success.
    
    I would like to acknowledge the funding received from the Productivity
    Partnership as supported by the Social Sciences and Humanities Research
    Council of Canada. It made it possible to access the data that was
    critical in the analysis for Chapter 2.
    
    Special thanks to others who have offered advice throughout my years as
    a doctoral student. In no particular order, Professors Svetlana
    Demidova, Cesar Sosa-Padilla, Arthur Sweetman, Alok Johri, Jeff Racine,
    Bettina Brueggemann, Stephen Jones and Phil DeCicca. As well as students
    Khuong Trong (the real HLP), Karen Bravo, Mahbub Rahman, Terry Yip,
    Muhebullah Karimzada, Shaun Shaikh, Farhana Khanam, Sean Sexton, Luc
    Clair, Grant Gibson, John Kealey, Natalie Malak, Nadine Chami, Anthony
    Hong and Zvez Todorov. A special thank to James Spencer who, among many
    other contributions, created this McMaster thesis template in \emph{R
    Markdown} allowing for seamless formatting of this document.
    
    Finally, I am thankful for the love and support of my parents and
    grandfather who always encouraged my academic endeavours. Also, to my
    aunts and uncles who have played a role in my life; Lise, Monique,
    Jacques, Darcie, Danny and Lisa. And to my best friend Set.
    
    This thesis is dedicated to my late mother Ginette Houle who taught me
    about courage and tenacity and who supported my fascination with science
    and mathematics at a young age.
    \thispagestyle{plain}
  \end{acknowledgements}
  \hypersetup{linkcolor=black}
  \setcounter{tocdepth}{2}
  \tableofcontents
  \thispagestyle{plain}

  \listoftables
  \thispagestyle{plain}

  \listoffigures
  \thispagestyle{plain}
  \begin{declaration}
    \label{doaac} The content of the research in this thesis is composed of
    work by Stephanie Houle, Pau S. Pujolas and Michael R. Veall. Chapters 2
    and 3 are solely the work of Stephanie Houle while Chapter 4 was joint
    work with Pau S. Pujolas and Michael R. Veall. To Chapter 4, I
    contributed all the data work, tables and graphs, and the writing for
    Section 4.5.
    \thispagestyle{plain}
  \end{declaration}


\mainmatter % here the regular arabic numbering starts
%\pagestyle{fancyplain} % turns page numbering back on




% Index?

\end{document}
